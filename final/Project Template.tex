
%% bare_conf.tex
%% V1.4b
%% 2015/08/26
%% by Michael Shell
%% See:
%% http://www.michaelshell.org/
%% for current contact information.
%%
%% This is a skeleton file demonstrating the use of IEEEtran.cls
%% (requires IEEEtran.cls version 1.8b or later) with an IEEE
%% conference paper.
%%
%% Support sites:
%% http://www.michaelshell.org/tex/ieeetran/
%% http://www.ctan.org/pkg/ieeetran
%% and
%% http://www.ieee.org/

%%*************************************************************************ww
%% Legal Notice:
%% This code is offered as-is without any warranty either expressed or
%% implied; without even the implied warranty of MERCHANTABILITY or
%% FITNESS FOR A PARTICULAR PURPOSE! 
%% User assumes all risk.
%% In no event shall the IEEE or any contributor to this code be liable for
%% any damages or losses, including, but not limited to, incidental,
%% consequential, or any other damages, resulting from the use or misuse
%% of any information contained here.
%%
%% All comments are the opinions of their respective authors and are not
%% necessarily endorsed by the IEEE.
%%
%% This work is distributed under the LaTeX Project Public License (LPPL)
%% ( http://www.latex-project.org/ ) version 1.3, and may be freely used,
%% distributed and modified. A copy of the LPPL, version 1.3, is included
%% in the base LaTeX documentation of all distributions of LaTeX released
%% 2003/12/01 or later.
%% Retain all contribution notices and credits.
%% ** Modified files should be clearly indicated as such, including  **
%% ** renaming them and changing author support contact information. **
%%*************************************************************************


% *** Authors should verify (and, if needed, correct) their LaTeX system  ***
% *** with the testflow diagnostic prior to trusting their LaTeX platform ***
% *** with production work. The IEEE's font choices and paper sizes can   ***
% *** trigger bugs that do not appear when using other class files.       ***                          ***
% The testflow support page is at:
% http://www.michaelshell.org/tex/testflow/



\documentclass[conference]{IEEEtran}
% Some Computer Society conferences also require the compsoc mode option,
% but others use the standard conference format.
%
% If IEEEtran.cls has not been installed into the LaTeX system files,
% manually specify the path to it like:
% \documentclass[conference]{../sty/IEEEtran}





% Some very useful LaTeX packages include:
% (uncomment the ones you want to load)


% *** MISC UTILITY PACKAGES ***
%
%\usepackage{ifpdf}
% Heiko Oberdiek's ifpdf.sty is very useful if you need conditional
% compilation based on whether the output is pdf or dvi.
% usage:
% \ifpdf
%   % pdf code
% \else
%   % dvi code
% \fi
% The latest version of ifpdf.sty can be obtained from:
% http://www.ctan.org/pkg/ifpdf
% Also, note that IEEEtran.cls V1.7 and later provides a builtin
% \ifCLASSINFOpdf conditional that works the same way.
% When switching from latex to pdflatex and vice-versa, the compiler may
% have to be run twice to clear warning/error messages.






% *** CITATION PACKAGES ***
%
%\usepackage{cite}
% cite.sty was written by Donald Arseneau
% V1.6 and later of IEEEtran pre-defines the format of the cite.sty package
% \cite{} output to follow that of the IEEE. Loading the cite package will
% result in citation numbers being automatically sorted and properly
% "compressed/ranged". e.g., [1], [9], [2], [7], [5], [6] without using
% cite.sty will become [1], [2], [5]--[7], [9] using cite.sty. cite.sty's
% \cite will automatically add leading space, if needed. Use cite.sty's
% noadjust option (cite.sty V3.8 and later) if you want to turn this off
% such as if a citation ever needs to be enclosed in parenthesis.
% cite.sty is already installed on most LaTeX systems. Be sure and use
% version 5.0 (2009-03-20) and later if using hyperref.sty.
% The latest version can be obtained at:
% http://www.ctan.org/pkg/cite
% The documentation is contained in the cite.sty file itself.






% *** GRAPHICS RELATED PACKAGES ***
%
\ifCLASSINFOpdf
  % \usepackage[pdftex]{graphicx}
  % declare the path(s) where your graphic files are
  % \graphicspath{{../pdf/}{../jpeg/}}
  % and their extensions so you won't have to specify these with
  % every instance of \includegraphics
  % \DeclareGraphicsExtensions{.pdf,.jpeg,.png}
\else
  % or other class option (dvipsone, dvipdf, if not using dvips). graphicx
  % will default to the driver specified in the system graphics.cfg if no
  % driver is specified.
  % \usepackage[dvips]{graphicx}
  % declare the path(s) where your graphic files are
  % \graphicspath{{../eps/}}
  % and their extensions so you won't have to specify these with
  % every instance of \includegraphics
  % \DeclareGraphicsExtensions{.eps}
\fi
% graphicx was written by David Carlisle and Sebastian Rahtz. It is
% required if you want graphics, photos, etc. graphicx.sty is already
% installed on most LaTeX systems. The latest version and documentation
% can be obtained at: 
% http://www.ctan.org/pkg/graphicx
% Another good source of documentation is "Using Imported Graphics in
% LaTeX2e" by Keith Reckdahl which can be found at:
% http://www.ctan.org/pkg/epslatex
%
% latex, and pdflatex in dvi mode, support graphics in encapsulated
% postscript (.eps) format. pdflatex in pdf mode supports graphics
% in .pdf, .jpeg, .png and .mps (metapost) formats. Users should ensure
% that all non-photo figures use a vector format (.eps, .pdf, .mps) and
% not a bitmapped formats (.jpeg, .png). The IEEE frowns on bitmapped formats
% which can result in "jaggedy"/blurry rendering of lines and letters as
% well as large increases in file sizes.
%
% You can find documentation about the pdfTeX application at:
% http://www.tug.org/applications/pdftex





% *** MATH PACKAGES ***
%
%\usepackage{amsmath}
% A popular package from the American Mathematical Society that provides
% many useful and powerful commands for dealing with mathematics.
%
% Note that the amsmath package sets \interdisplaylinepenalty to 10000
% thus preventing page breaks from occurring within multiline equations. Use:
%\interdisplaylinepenalty=2500
% after loading amsmath to restore such page breaks as IEEEtran.cls normally
% does. amsmath.sty is already installed on most LaTeX systems. The latest
% version and documentation can be obtained at:
% http://www.ctan.org/pkg/amsmath





% *** SPECIALIZED LIST PACKAGES ***
%
%\usepackage{algorithmic}
% algorithmic.sty was written by Peter Williams and Rogerio Brito.
% This package provides an algorithmic environment fo describing algorithms.
% You can use the algorithmic environment in-text or within a figure
% environment to provide for a floating algorithm. Do NOT use the algorithm
% floating environment provided by algorithm.sty (by the same authors) or
% algorithm2e.sty (by Christophe Fiorio) as the IEEE does not use dedicated
% algorithm float types and packages that provide these will not provide
% correct IEEE style captions. The latest version and documentation of
% algorithmic.sty can be obtained at:
% http://www.ctan.org/pkg/algorithms
% Also of interest may be the (relatively newer and more customizable)
% algorithmicx.sty package by Szasz Janos:
% http://www.ctan.org/pkg/algorithmicx




% *** ALIGNMENT PACKAGES ***
%
%\usepackage{array}
% Frank Mittelbach's and David Carlisle's array.sty patches and improves
% the standard LaTeX2e array and tabular environments to provide better
% appearance and additional user controls. As the default LaTeX2e table
% generation code is lacking to the point of almost being broken with
% respect to the quality of the end results, all users are strongly
% advised to use an enhanced (at the very least that provided by array.sty)
% set of table tools. array.sty is already installed on most systems. The
% latest version and documentation can be obtained at:
% http://www.ctan.org/pkg/array


% IEEEtran contains the IEEEeqnarray family of commands that can be used to
% generate multiline equations as well as matrices, tables, etc., of high
% quality.




% *** SUBFIGURE PACKAGES ***
%\ifCLASSOPTIONcompsoc
%  \usepackage[caption=false,font=normalsize,labelfont=sf,textfont=sf]{subfig}
%\else
%  \usepackage[caption=false,font=footnotesize]{subfig}
%\fi
% subfig.sty, written by Steven Douglas Cochran, is the modern replacement
% for subfigure.sty, the latter of which is no longer maintained and is
% incompatible with some LaTeX packages including fixltx2e. However,
% subfig.sty requires and automatically loads Axel Sommerfeldt's caption.sty
% which will override IEEEtran.cls' handling of captions and this will result
% in non-IEEE style figure/table captions. To prevent this problem, be sure
% and invoke subfig.sty's "caption=false" package option (available since
% subfig.sty version 1.3, 2005/06/28) as this is will preserve IEEEtran.cls
% handling of captions.
% Note that the Computer Society format requires a larger sans serif font
% than the serif footnote size font used in traditional IEEE formatting
% and thus the need to invoke different subfig.sty package options depending
% on whether compsoc mode has been enabled.
%
% The latest version and documentation of subfig.sty can be obtained at:
% http://www.ctan.org/pkg/subfig




% *** FLOAT PACKAGES ***
%
%\usepackage{fixltx2e}
% fixltx2e, the successor to the earlier fix2col.sty, was written by
% Frank Mittelbach and David Carlisle. This package corrects a few problems
% in the LaTeX2e kernel, the most notable of which is that in current
% LaTeX2e releases, the ordering of single and double column floats is not
% guaranteed to be preserved. Thus, an unpatched LaTeX2e can allow a
% single column figure to be placed prior to an earlier double column
% figure.
% Be aware that LaTeX2e kernels dated 2015 and later have fixltx2e.sty's
% corrections already built into the system in which case a warning will
% be issued if an attempt is made to load fixltx2e.sty as it is no longer
% needed.
% The latest version and documentation can be found at:
% http://www.ctan.org/pkg/fixltx2e


%\usepackage{stfloats}
% stfloats.sty was written by Sigitas Tolusis. This package gives LaTeX2e
% the ability to do double column floats at the bottom of the page as well
% as the top. (e.g., "\begin{figure*}[!b]" is not normally possible in
% LaTeX2e). It also provides a command:
%\fnbelowfloat
% to enable the placement of footnotes below bottom floats (the standard
% LaTeX2e kernel puts them above bottom floats). This is an invasive package
% which rewrites many portions of the LaTeX2e float routines. It may not work
% with other packages that modify the LaTeX2e float routines. The latest
% version and documentation can be obtained at:
% http://www.ctan.org/pkg/stfloats
% Do not use the stfloats baselinefloat ability as the IEEE does not allow
% \baselineskip to stretch. Authors submitting work to the IEEE should note
% that the IEEE rarely uses double column equations and that authors should try
% to avoid such use. Do not be tempted to use the cuted.sty or midfloat.sty
% packages (also by Sigitas Tolusis) as the IEEE does not format its papers in
% such ways.
% Do not attempt to use stfloats with fixltx2e as they are incompatible.
% Instead, use Morten Hogholm'a dblfloatfix which combines the features
% of both fixltx2e and stfloats:
%
% \usepackage{dblfloatfix}
% The latest version can be found at:
% http://www.ctan.org/pkg/dblfloatfix




% *** PDF, URL AND HYPERLINK PACKAGES ***
%
%\usepackage{url}
% url.sty was written by Donald Arseneau. It provides better support for
% handling and breaking URLs. url.sty is already installed on most LaTeX
% systems. The latest version and documentation can be obtained at:
% http://www.ctan.org/pkg/url
% Basically, \url{my_url_here}.



 
% *** Do not adjust lengths that control margins, column widths, etc. ***
% *** Do not use packages that alter fonts (such as pslatex).         ***
% There should be no need to do such things with IEEEtran.cls V1.6 and later.
% (Unless specifically asked to do so by the journal or conference you plan
% to submit to, of course. )


% correct bad hyphenation here
\hyphenation{op-tical net-works semi-conduc-tor}


\begin{document}
%
% paper title
% Titles are generally capitalized except for words such as a, an, and, as,
% at, but, by, for, in, nor, of, on, or, the, to and up, which are usually
% not capitalized unless they are the first or last word of the title.
% Linebreaks \\ can be used within to get better formatting as desired.
% Do not put math or special symbols in the title.
\title{Review of Clinical Decision Support and Informatics applications in disease diagnosis applications in disease diagnosis}


% author names and affiliations
% use a multiple column layout for up to three different
% affiliations
\author{\IEEEauthorblockN{Sams Khan}
\IEEEauthorblockA{College of Computing\\
Kennesaw State University\\
Marietta, Georgia 30060\\
Email: ssamskhann@gmail.com}
}

% conference papers do not typically use \thanks and this command
% is locked out in conference mode. If really needed, such as for
% the acknowledgment of grants, issue a \IEEEoverridecommandlockouts
% after \documentclass

% for over three affiliations, or if they all won't fit within the width
% of the page, use this alternative format:
% 
%\author{\IEEEauthorblockN{Michael Shell\IEEEauthorrefmark{1},
%Homer Simpson\IEEEauthorrefmark{2},
%James Kirk\IEEEauthorrefmark{3}, 
%Montgomery Scott\IEEEauthorrefmark{3} and
%Eldon Tyrell\IEEEauthorrefmark{4}}
%\IEEEauthorblockA{\IEEEauthorrefmark{1}School of Electrical and Computer Engineering\\
%Georgia Institute of Technology,
%Atlanta, Georgia 30332--0250\\ Email: see http://www.michaelshell.org/contact.html}
%\IEEEauthorblockA{\IEEEauthorrefmark{2}Twentieth Century Fox, Springfield, USA\\
%Email: homer@thesimpsons.com}
%\IEEEauthorblockA{\IEEEauthorrefmark{3}Starfleet Academy, San Francisco, California 96678-2391\\
%Telephone: (800) 555--1212, Fax: (888) 555--1212}
%\IEEEauthorblockA{\IEEEauthorrefmark{4}Tyrell Inc., 123 Replicant Street, Los Angeles, California 90210--4321}}




% use for special paper notices
%\IEEEspecialpapernotice{(Invited Paper)}




% make the title area
\maketitle

% As a general rule, do not put math, special symbols or citations
% in the abstract
\begin{abstract}
Diagnosis of a disease and classifying certain anomalies is one of the most important jobs, doctors have to do and hope to do as accurately as possible. Certain diseases and anomalies are hard to diagnose due to ambiguous symptoms, test results and personal experience of the physician. Being able to automate this task with relatively high accuracy in one of the biggest leaps in medical technology we can hope for. In this paper we review some papers that try apply computational methods to automate and accurately diagnose certain diseases and anomalies. The 2 specific methods highlighted in this paper are the usage of CNNs(Convolutional Neural Networks) which yielded highly accurate results from extracting features and classifying image data.
\end{abstract}

% no keywords




% For peer review papers, you can put extra information on the cover
% page as needed:
% \ifCLASSOPTIONpeerreview
% \begin{center} \bfseries EDICS Category: 3-BBND \end{center}
% \fi
%
% For peerreview papers, this IEEEtran command inserts a page break and
% creates the second title. It will be ignored for other modes.
\IEEEpeerreviewmaketitle



\section{Introduction}

Clinical Decision Support and Informatics is a broad umbrella of research, the goal is to create  applications or tools that can be used to automate tasks in clinical biology or help professionals in making decisions and collecting information. In this paper i will review different papers that propose multiple methods to diagnose patients of certain diseases. 
Doctors usually make a diagnosis of a patient by arriving at conclusions based on some clinical evidences such as symptoms, different signs and even their own clinical experience. But to attain such features in a computational aspect is quite challenging because there are just so many cases that it must go through to make a diagnosis like a doctor does.




\section{Related Works}
\textit{A. CNN-based image analysis for malaria diagnosis}
Malaria is a major global threat and the standard way of diagnosing is by visually examining blood smears for parasite infected blood under a microscope. In this paper proposes a robust solution using Convolutional Neural Network/(CNN) to automatically classify blood samples as infected or uninfected.
This method has high computational cost and takes long time to train the model.

\textit{B. Biomedical Image analysis for early diagnosis of Breast Cancer}
This paper discusses and compares 6 different types of classification algorithms that can early diagnose breast cancer. 
It isn't a normal paper where it just talks about one method but compares and contrasts 6 different ones and talks about which one is the best for microarray based data-sets and image based datasets.

\textit{C. Knowledge acquisition for medical diagnosis using collective intelligence }
The wisdom of the crowds is when you take into account a collective opinion of a group instead of a single expert. This paper uses a technique similar to this to collect diagnosis information in Diagnosis Decision Support System. Since a lot of the times diagnosis of a patient not only requires knowing what to look for and what symptoms they may be playing. A professional's personal opinion and experience plays a key role into making this decision.
The main problem behind the scenario in this paper is that even when the diagnostic criteria can change from one
physician to another, they are not sure if these changes
are enough to imply a sufficient change in the modeling of
the diagnostic criteria.

\textit{D. Intelligent Diagnosis Method of Cardiovascular Anomal ies Using Medical Signal Processing }
The goal of this work is to make an automatic diagnosis based on the ICG corresponding to the Aorta impedance variation and ECG signal during the heart cycle activity. This entire process is designed to be non-invasive and automatic diagnosis using a graphical UI made using MATLAB. Biggest limitation to this method is that it requires a database of different cardiac diseases and anomalies and their ICG and ECG data.

\textit{E. Localization and diagnosis framework for pediatric cataracts based on slit-lamp images using deep features of a Convolution Neural Network }
Through the implementation of deep learning convolutional neural networks in ophthalmology, it is possible to detect and examine pediatric cataracts based on slit-lamp images. The referenced model provides a potentially automatic diagnosis utility for ophthalmologists which would not only save time and energy but also allow for quicker treatment options. Based on current observations, the deep learning model performed exceptionally well in both accuracy and efficiency compared to traditional diagnosis methods. Furthermore, the CNN is capable of self-corrections and is constantly improving its reliability. However, one limitation is the training of the CNN using labeled cataract image data is very computationally heavy and may take a long time. This will also require the use of a very powerful GPU.

\textit{F. Pre-Trained Convolutional Neural Network Based Method for Thyroid Nodule Diagnosis}
Pre-trained convolutional neural networks present a new clinical method for the diagnosis of thyroid nodules. The aforementioned process revolves around the fusion of two separately trained convolutional networks which would be integrated with layered filters and feature maps. The convolutional neural networks allows for a “real-time and non-invasive” administration and performed with a diagnostic accuracy of ~83.02%. The current traditionally accepted method of diagnosing thyroid nodules requires the use of biopsies which are extremely labor intensive and when used unnecessarily “make patients more anxious and increase the health care costs”. One such limitation is the generalization of the CNN to the two local hospitals

\section{Methods}


\subsubsection{Method 1: C. Knowledge acquisition for medical diagnosis using collective intelligence }

	Each physician has their own knowledge base that they can make changes to and put in different criteria for diagnosing disease. We then use this data to build the consensus knowledge base and collective intelligence.

	Consensus process 1: Signs coincidence
	The first process takes into account the expert's coincidence in the findings, so it finds things that are the same between multiple physicians. This is based on the percentage scale, so only findings in the 70\% or more of the cases are taken into account for building the consensus ontology. 

	Consensus process 2: Pairwise similarity

	The second method makes comparisons between pairs of individual knowledge bases bearing in mind that the pairs have the highest level of similarity. This is done by:
	1) Making the union of all of them.
	2) Making the intersection of them.

	They use this to find the biggest value and then use the opinion that has the largest value similarity and then adds it to the consensus knowledge base.

	Consensus process 3: Modification ranking

	This counts he modification each physician makes to their personal knowledge base, to do this they use 2 methods: Global and local methods. In the global method, weights are assigned to each expert physician.
	The local method is similar except that the weights are assigned to the findings themselves rather than the diseases, and this is repeated for each disease.

	The evaluation of the consensus methods is performed by comparing the diagnosis result taken from the consensus knowledge base. PRAS \textit{Precision , Recall, Accuracy, Specify and MCC} metrics are used to evaluate several aspects of the resulting knowledge base. The paper talks about using the pre-made evaluation method listen in the paper's sources.

	Pros: This method can vastly improve the accuracy of a particular knowledge base and can represent the opinions and personal experience of multiple physicians in a single place.

	Cons: Diagnostic criteria can change from physician to physician, and the modification made by each physicians, we don't know for sure if this is enough to imply a significant change in the modeling of the database. 

\subsubsection{Method 2: E. Localization and diagnosis framework for pediatric cataracts based on slit-lamp images using deep features of a Convolution Neural Network\\}
\subsection{Slit lamp photography method}
A high-intensity light source instrument, is used to shine a thin sheet of light into the eye to
examine the anterior segment and posterior segment of the human eye.
Slit lamp photography image data was used to train the CNN.

\subsection{Diagnosis Framework}
The framework offers 3 parts, automatic localization for lens, classification and three-degree grading. First, using Candy	detection and Hough transformation the lens ROI is localized.Then it is fed into the CNN to extract high level features and apply classification and grading. 

\subsection{Deep Convolutional neural network}
The overall architecture of the CNN used in this method has 5 convolution and overlapping max pooling layers followed by 3 fully connected layers. Th first 7 layers are used to extract multidimensional and high-level features from the input image, and the softmax layer is applied to classification and grading.

The assumption for this method was to accurately detect cataracts using the CNN and then classify it.

Pros: Highly accurate, this method and usage of CNN's and classification had high results of accuracy and level of classification.

Cons: Required a long time to train, this neural net took a long time to train and was very computationally heavy, it required a NVDIA GTX Titan x to train.\\
\subsubsection{Method 3: F. Pre-Trained Convolutional Neural Network Based Method for Thyroid Nodule Diagnosis}
\subsection{Image Aquisition and Pre-processing}
8148 anonymous thyroid nodule ultrasounds were collected from 4782 patients who have gone through surgeries. These were used as ground truth. 15000 Thyroid nodule pre-surgery images were taken from multiple systems and were used in training the CNN's.\\

\subsection{Architectures of CNNs}
In this study a combination of 2 CNNs were used. The first CNN was based on the CADx framework, this was used for feature learning and classification. In feature learning it uses 3 convolutional layers with normalization and pooling functions to extract features, then for classification it uses Softmax with the fully connected layer feature map the thyroid nodules. The second CNN had 5 convolutional layers and 3 fully connected layers, and the other layers are the same as the first CNN.\\
The point of having 2 different CNNs with 2 different architectures were because they can learn different features. The first network can learn the low level features and the 2nd deep network can learn all the high level features. The features are then fused by a sumlayer and trains a softmax layer. 

Because of the low quality of the ultrasound images, it makes it really hard to properly classify thyroid nodules. By using this implementation the researchers successfully managed to classify the the ultrasound images to a reputable degree.

Pros: Highly accurate results and classification, to make it successfully automate the task of diagnosing malignant thyroid nodules.

Cons: This paper does not list many cons to their method. 


% An example of a floating figure using the graphicx package.
% Note that \label must occur AFTER (or within) \caption.
% For figures, \caption should occur after the \includegraphics.
% Note that IEEEtran v1.7 and later has special internal code that
% is designed to preserve the operation of \label within \caption
% even when the captionsoff option is in effect. However, because
% of issues like this, it may be the safest practice to put all your
% \label just after \caption rather than within \caption{}.
%
% Reminder: the "draftcls" or "draftclsnofoot", not "draft", class
% option should be used if it is desired that the figures are to be
% displayed while in draft mode.
%
%\begin{figure}[!t]
%\centering
%\includegraphics[width=2.5in]{myfigure}
% where an .eps filename suffix will be assumed under latex, 
% and a .pdf suffix will be assumed for pdflatex; or what has been declared
% via \DeclareGraphicsExtensions.
%\caption{Simulation results for the network.}
%\label{fig_sim}
%\end{figure}

% Note that the IEEE typically puts floats only at the top, even when this
% results in a large percentage of a column being occupied by floats.


% An example of a double column floating figure using two subfigures.
% (The subfig.sty package must be loaded for this to work.)
% The subfigure \label commands are set within each subfloat command,
% and the \label for the overall figure must come after \caption.
% \hfil is used as a separator to get equal spacing.
% Watch out that the combined width of all the subfigures on a 
% line do not exceed the text width or a line break will occur.
%
%\begin{figure*}[!t]
%\centering
%\subfloat[Case I]{\includegraphics[width=2.5in]{box}%
%\label{fig_first_case}}
%\hfil
%\subfloat[Case II]{\includegraphics[width=2.5in]{box}%
%\label{fig_second_case}}
%\caption{Simulation results for the network.}
%\label{fig_sim}
%\end{figure*}
%
% Note that often IEEE papers with subfigures do not employ subfigure
% captions (using the optional argument to \subfloat[]), but instead will
% reference/describe all of them (a), (b), etc., within the main caption.
% Be aware that for subfig.sty to generate the (a), (b), etc., subfigure
% labels, the optional argument to \subfloat must be present. If a
% subcaption is not desired, just leave its contents blank,
% e.g., \subfloat[].


% An example of a floating table. Note that, for IEEE style tables, the
% \caption command should come BEFORE the table and, given that table
% captions serve much like titles, are usually capitalized except for words
% such as a, an, and, as, at, but, by, for, in, nor, of, on, or, the, to
% and up, which are usually not capitalized unless they are the first or
% last word of the caption. Table text will default to \footnotesize as
% the IEEE normally uses this smaller font for tables.
% The \label must come after \caption as always.
%
%\begin{table}[!t]
%% increase table row spacing, adjust to taste
%\renewcommand{\arraystretch}{1.3}
% if using array.sty, it might be a good idea to tweak the value of
% \extrarowheight as needed to properly center the text within the cells
%\caption{An Example of a Table}
%\label{table_example}
%\centering
%% Some packages, such as MDW tools, offer better commands for making tables
%% than the plain LaTeX2e tabular which is used here.
%\begin{tabular}{|c||c|}
%\hline
%One & Two\\
%\hline
%Three & Four\\
%\hline
%\end{tabular}
%\end{table}


% Note that the IEEE does not put floats in the very first column
% - or typically anywhere on the first page for that matter. Also,
% in-text middle ("here") positioning is typically not used, but it
% is allowed and encouraged for Computer Society conferences (but
% not Computer Society journals). Most IEEE journals/conferences use
% top floats exclusively. 
% Note that, LaTeX2e, unlike IEEE journals/conferences, places
% footnotes above bottom floats. This can be corrected via the
% \fnbelowfloat command of the stfloats package.




\section{Conclusion}
The challenges in automating diagnosis pose a plethora of challenges. Usage of similar methods mentioned in this paper and much more can let us get closer and closer to having a much more accurate way of helping doctors diagnose patients. The methods listed above have a lot of limitations, but the results are still promising and can be used in this area of research. The usage of modern methods of deep learning shown in papers E and F show very little limitation once the network is trained on accurate labeled data and the results are in the high upper bounds. The objective of the paper was to review different ways of diagnosing disease and different methods of helping physicians do it, and based on the  papers reviewed in this paper, the papers with the most future prospects and highest levels of accuracy is the methods that used Convolution Neural Networks. 




% conference papers do not normally have an appendix


% use section* for acknowledgment
%\section*{Acknowledgment}


%The authors would like to thank...





% trigger a \newpage just before the given reference
% number - used to balance the columns on the last page
% adjust value as needed - may need to be readjusted if
% the document is modified later
%\IEEEtriggeratref{8}
% The "triggered" command can be changed if desired:
%\IEEEtriggercmd{\enlargethispage{-5in}}

% references section

% can use a bibliography generated by BibTeX as a .bbl file
% BibTeX documentation can be easily obtained at:
% http://mirror.ctan.org/biblio/bibtex/contrib/doc/
% The IEEEtran BibTeX style support page is at:
% http://www.michaelshell.org/tex/ieeetran/bibtex/
%\bibliographystyle{IEEEtran}
% argument is your BibTeX string definitions and bibliography database(s)
%\bibliography{IEEEabrv,../bib/paper}
%
% <OR> manually copy in the resultant .bbl file
% set second argument of \begin to the number of references
% (used to reserve space for the reference number labels box)
\begin{thebibliography}{1}

\bibitem{promoters}
Junbing, Li. "Application of BP Neural Network Algorithm in Biomedical Diagnostic Analysis." International Journal Bioautomation, vol. 20, no. 3, July 2016, pp. 417-426. EBSCOhost, proxy.kennesaw.edu/login?url=http://search.ebscohost.com/login.aspx?direct=true\&db=a9h\&AN=119098971\&site=eds-live\&scope=site.

\bibitem{malaria}
CNN-based image analysis for malaria diagnosis. (2016). 2016 IEEE International Conference on Bioinformatics and Biomedicine (BIBM), Bioinformatics and Biomedicine (BIBM), 2016 IEEE International Conference on, 493. doi:10.1109/BIBM.2016.7822567

\bibitem{breast cancer}
Nahar, Jesmin, et al. "Computational Intelligence for Microarray Data and Biomedical Image Analysis for the Early Diagnosis of Breast Cancer." Expert Systems with Applications, vol. 39, 15 Nov. 2012, pp. 12371-12377. EBSCOhost, doi:10.1016/j.eswa.2012.04.045.

\bibitem{intelligence}
Hernández-Chan, G., Rodríguez-González, A., Alor-Hernández, G., Gómez-Berbís, J. M., Mayer-Pujadas, M. A., \& Posada-Gómez, R. (2012). Knowledge acquisition for medical diagnosis using collective intelligence. Journal Of Medical Systems, 36 Suppl 1S5-S9. doi:10.1007/s10916-012-9886-3

\bibitem{signal_process}
"Intelligent Diagnosis Method of Cardiovascular Anomalies Using Medical Signal Processing." 2015 World Congress on Information Technology and Computer Applications (WCITCA), Information Technology and Computer Applications Congress (WCITCA), 2015 World Congress on, 2015, p. 1. EBSCOhost, doi:10.1109/WCITCA.2015.7367032.

\bibitem{Survival}
Liu, X., Jiang, J., Zhang, K., Long, E., Cui, J., Zhu, M., \& ... Lin, H. (2017). Localization and diagnosis framework for pediatric cataracts based on slit-lamp images using deep features of a convolutional neural network. Plos ONE, 12(3), 1-18. doi:10.1371/journal.pone.0168606

\bibitem{Epilepsy}
Improving convolutional neural network using accelerated proximal gradient method for epilepsy diagnosis. (2016). 2016 UKACC 11th International Conference on Control (CONTROL), Control (CONTROL), 2016 UKACC 11th International Conference on, 1. doi:10.1109/CONTROL.2016.7737620

\bibitem{Thyroid}
Ma, Jinlian, et al. "A Pre-Trained Convolutional Neural Network Based Method for Thyroid Nodule Diagnosis." Ultrasonics, vol. 73, 01 Jan. 2017, pp. 221-230. EBSCOhost, doi:10.1016/j.ultras.2016.09.011.

\end{thebibliography}




% that's all folks
\end{document}


